% Options for packages loaded elsewhere
\PassOptionsToPackage{unicode}{hyperref}
\PassOptionsToPackage{hyphens}{url}
%
\documentclass[
]{article}
\usepackage{amsmath,amssymb}
\usepackage{iftex}
\ifPDFTeX
  \usepackage[T1]{fontenc}
  \usepackage[utf8]{inputenc}
  \usepackage{textcomp} % provide euro and other symbols
\else % if luatex or xetex
  \usepackage{unicode-math} % this also loads fontspec
  \defaultfontfeatures{Scale=MatchLowercase}
  \defaultfontfeatures[\rmfamily]{Ligatures=TeX,Scale=1}
\fi
\usepackage{lmodern}
\ifPDFTeX\else
  % xetex/luatex font selection
\fi
% Use upquote if available, for straight quotes in verbatim environments
\IfFileExists{upquote.sty}{\usepackage{upquote}}{}
\IfFileExists{microtype.sty}{% use microtype if available
  \usepackage[]{microtype}
  \UseMicrotypeSet[protrusion]{basicmath} % disable protrusion for tt fonts
}{}
\makeatletter
\@ifundefined{KOMAClassName}{% if non-KOMA class
  \IfFileExists{parskip.sty}{%
    \usepackage{parskip}
  }{% else
    \setlength{\parindent}{0pt}
    \setlength{\parskip}{6pt plus 2pt minus 1pt}}
}{% if KOMA class
  \KOMAoptions{parskip=half}}
\makeatother
\usepackage{xcolor}
\setlength{\emergencystretch}{3em} % prevent overfull lines
\providecommand{\tightlist}{%
  \setlength{\itemsep}{0pt}\setlength{\parskip}{0pt}}
\setcounter{secnumdepth}{-\maxdimen} % remove section numbering
\ifLuaTeX
  \usepackage{selnolig}  % disable illegal ligatures
\fi
\IfFileExists{bookmark.sty}{\usepackage{bookmark}}{\usepackage{hyperref}}
\IfFileExists{xurl.sty}{\usepackage{xurl}}{} % add URL line breaks if available
\urlstyle{same}
\hypersetup{
  pdftitle={Using economic theory discuss the determinants of GDP per capita in the long-run, and the growth rate of GDP per capita in the long-run. Examine how these two phenomena respond to a permanent decrease in the growth rate of the population.},
  hidelinks,
  pdfcreator={LaTeX via pandoc}}

\title{Using economic theory discuss the determinants of GDP per capita
in the long-run, and the growth rate of GDP per capita in the long-run.
Examine how these two phenomena respond to a permanent decrease in the
growth rate of the population.}
\author{}
\date{}

\begin{document}
\maketitle

To analyse determinants of GDP per capita and growth in the long run we
can use the Solow-Swan model. We can derive long run GDP using an
\textbf{aggregate production function} which describes the relationship
between production inputs and total output, {\(Y\)}. Since the main two
inputs in production are capital, {\(K\)}, and number of workers,
{\(N\)}, the aggregate production function can be described as:
{\[Y = F(K,N)\]}This relation can be amended to include technological
progress by adding a coefficient to the number of workers {\(A\)}, which
can be inferred as the \emph{given state of technology} within the
economy: {\[Y = F(K,AN)\]}The implication that follows is that for any
doubling of the state of technology has explicitly the same impact of
doubling the number of workers. In practice, this holds as innovation
that doubles the marginal output per worker would result in half as many
workers needed to make the same number of goods. {\(AN\)} henceforth
will be referred to as effective workers, as it implies the amount of
workers needed to produce the same output without current state of
technology: 1 worker with technology that doubles their output is
\emph{effectively} the same as 2 workers without said technology.

The aggregate production function can be manipulated to output per
capita if we assume the function will always have constant returns to
scale. In this case, any proportional change in both capital and labour
would result in the same proportional change in output. For example, a
doubling of both capital and number of workers would result in a
doubling of output. This behaviour can be described as,
{\[xY = F(xK,xAN)\]}Since our number of effective workers is fixed at
any point in time, we can set {\(x\)} to the reciprocal of the number of
effective workers, {\(\frac{1}{AN}\)} and thus derive the output per
capita production function:

\[\begin{matrix}
\frac{Y}{AN} & {= F\left( \frac{K}{AN},\frac{AN}{AN} \right)} \\
 & {= F\left( \frac{K}{AN},1 \right)} \\
 & {= f\left( \frac{K}{AN} \right)} \\
\end{matrix}\]

Since, the aggregate production function is essentially divided by the
number of effective workers, {\(AN\)} itself becomes a constant since
{\(\frac{AN}{AN}\)} is 1, and thus not a determinant in the output per
capita aggregate production function. For simplicity, this relationship
can be expressed as function {\(f\left( \frac{K}{AN} \right)\)}. To
establish an equilibrium output per worker (the steady-state output) we
must determine the factors that change capital per worker over a period
of time.

Allowing for technological progress, the number of effective workers
increases over time. Consequently, to retain the existing ratio of
capital to effective workers, {\(\frac{K}{AN}\)}, an increase in the
capital stock {\(K\)} proportional to an increase in the number of
effective workers {\(AN\)} is needed. Furthermore, the capital stock
depreciates over time as the value of machinery decreases and repairs
need to be made. Therefore, additional investment on top of that needed
to retain the capital to effective worker ratio is needed to prevent the
value of capital stock deteriorating over time due to depreciation.
Also, the growth rate of workers impacts the ratio of capital to
effective worker; if over a period of time the number of workers
decrease, then less investment is required to retain the current ratio
of capital to effective worker. Combining these determinants, the value
of future capital stock {\(K_{t + 1}\)} can be derived:
{\[K_{t + 1} = (1 - \delta + g_{N} + g_{A})K_{t} + I_{t}\]}To derive
changes in capital stock over time, {\(K_{t + 1} - K_{t}\)}, we can
subtract {\(K_{t}\)} from both sides of the relation:
{\[K_{t + 1} - K_{t} = (g_{N} + g_{A} - \delta)K_{t} + I_{t}\]}To bridge
the gap between changes in capital stock and output in the long run, the
investment-savings relation can be used, whereby current investment
{\(I_{t}\)} is equal to savings: the savings rate {\(S\)} multiplied by
output {\(Y\)}. {\[I_{t} = sY_{t}\]}Inserting this into the capital
stock relation,
{\[K_{t + 1} = (1 - \delta + g_{N} + g_{A})K_{t} + sY_{t}\]}

\end{document}
